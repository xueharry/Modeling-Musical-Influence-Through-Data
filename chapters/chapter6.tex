\begin{savequote}[75mm] 
Only hope that we kinda have left is music and vibrations\\
Lot a people don’t understand how important it is, you know
\qauthor{``Mortal Man''- Kendrick Lamar} 
\end{savequote}

\chapter{Conclusion}

In this thesis we investigated modeling musical influence through data, settling on an audio content-based approach using 143,625 audio files and a ground truth human expert curated network graph of artist-to-artist influence consisting of 16,704 artists scraped from AllMusic.com. We first tackled this problem through a topic modeling approach, using the Document Influence Model to find a significant correlation with node outdegree in our ground truth graph. Due to a need for richer feature representation and a desire to classify artist-to-artist influence, we proposed a novel approach using siamese convolutional neural networks, achieving a validation accuracy of 0.7 on predicting binary influence between 3 second mel-spectogram samples from pairs of input songs. Our method perhaps represents the most general attempt at modeling musical influence to date; we make no assumptions about the definition of influence, having the model learn to discriminate influence based on labeled examples and our model is easily extensible to song pairs (and hence influence relationships) not seen in training as opposed to having a fixed number of class labels. Additionally our method is extensible for use in other musical influence related applications such as relative ranking of influence strength as shown by the ranking algorithm we proposed.

What else can our model be used for? From a knowledge discovery perspective, it could be used to discover new influence relationships between music artists in a data-driven way. From a practical perspective, given the massive volume of music available for listeners today through services such as Apple Music, Spotify and Pandora (to name a few), there exists a need for ways of cataloging and organizing it all. Influence perhaps represents one such way. For instance, one can reasonably imagine a music recommendation system that incorporates influence information in curating playlists for listeners. In addition, with appropriate feature representations, our approach could be generalized for modeling influence in other forms of media, such as speech audio or text.

Our research has also raised many more questions and reflects broader issues beyond the fairly narrow scope of our work. For example, what is the best way to represent audio data? Admittingly, as we saw in this thesis, bag-of-words representations can be quite limited. We found good performance with applying deep learning methods to extract features from intermediate mel-spectrogram representations, but recently there have been beginning attempts to utilize raw waveforms directly which may serve as an interesting research direction \cite{gong2018how}. As another example, given examples of multiple songs for a given artist, how do we create an artist-level summary beyond just simple averaging? Reworded more generally (in natural language processing terms), given a corpus of multiple documents for an author, how might we create an author-level summary? These questions are not just limited to, and in fact extend well beyond the problem of modeling musical influence.