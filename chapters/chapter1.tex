\begin{savequote}[75mm] 
Then reach the stars, you take the time\\
To look behind and say, "Look where I came\\
Look how far I done came"
\qauthor{``Intro''- J. Cole} 
\end{savequote}

\chapter{Introduction}

\section{Motivation}
The study of musical influence relationships is a topic of great interest to music researchers, critics and general enthusiasts alike. As humans, we often use influence terminology in order to situate a musical artist on the sonic spectrum. For instance, music critics will often introduce a new artist in terms of their influences. As another example, a person recommending an artist to a friend will often speak of the musician in terms of \textit{who} he or she sounds like. 

Though pretty intuitive to humans, inferring influence relationships computationally is not a straightforward task. Part of the difficulty in modeling musical influence computationally arises from a lack of a precise definition for influence, making it a rather abstract task. Though most people have a good general sense of what it means for one musical artist to have influenced another, in reality influence can take on several different meanings. As Morton and Kim note \cite{morton2015acoustic}, one artist may have exerted direct influence on another artist through direct and prolonged personal interactions. These interactions include ``teacher-student relationships, band membership, frequent collaborations between artists, and even familial associations'' \cite{morton2015acoustic}. Not all influence relationships take on this sort of flavor however; many artists are voracious listeners themselves and will be influenced by something as brief as a 30-segment segment of a song that they happen to hear by chance while sitting in a coffee shop. The genre of hip-hop is a salient example of this, as producers and musicians will often use such ``found sounds'' as samples that are incorporated in the creation of new works. 

Influence also varies in the way it ultimately manifests itself in an artist's work. In the case of hip-hop, sometimes this can be rather obvious as an artist will directly sample a strong influence of theirs. In other instances however, detecting influential elements can be much more difficult. For example, a jazz musician might try to incorporate minutia such as the articulation patterns, harmonic vocabulary and/or timbre of an influence in their own playing. Such influence can be difficult to detect for even the human listener, requiring a keen ear and extended listening to unpack. Furthermore, it is also important to note the distinction between influence and similarity. Though artists who exhibit an influence relationship often will sound similar, this is not necessarily the case. As Morton and Kim point out, ``one artist may have had a large influence on one another and yet the two musicians differ greatly in terms of perception'' \cite{morton2015acoustic}. 

In addition to the challenge of a lack of a clear definition for influence, the problem of modeling influence also reflects broader challenges in the field of music information retrieval (MIR) in general, especially if one is take an audio-based approach. First, audio data is quite complex, containing rich structural information on multiple timescales and second, music itself is ever-evolving as artists, songs and genres all change over time \cite{shalit2013modeling}. With regard to dealing with the complexity of audio data, there still exists a large semantic gap in extracting high-level properties such as ``genre, mood, instrumentation and themes'' from audio  \cite{van2013deep}. In terms of the evolving nature of music, it is quite difficult to create models complex enough to capture these shifting relationships.

Despite these challenges, inferring musical influence surely is significant from a musicological and sociological standpoint. Influence relationships can help us better understand the historical development of genres and the overall evolution of music over time. For instance, why are certain musical elements more enduring than others and hence become more influential over time? Besides knowledge discovery however, inferring musical influence also has practical application. With today's vast quantity of available music metadata and music audio, which in and of itself is a form of data, there exists a need for new methods of cataloging and organizing it all. Influence relationships perhaps serve as one such means of making sense of this data.

The scale of data available today however has a silver lining though--- while creating new challenges, it also presents an opportunity to study music influence in a data-driven way that was not possible until recently. With the vast availability of album metadata, cover song listings, collaboration information, lyrics and song audio available on the internet, there are many plausible approaches to tackling the problem of modeling musical influence. Though far from being exhaustive, we explore several of these approaches in this thesis.

\section{Related Work}
Previous work has been done on analyzing known sample-based musical influence networks \cite{bryan2011musical}, but with the exception of the work of Collins \cite{collins2010computational, collins2012influence}Shalit et al. \cite{shalit2013modeling} and Morton et al.\cite{morton2015acoustic}, there has been limited work done on the task of inferring musical influence relationships through data.

Nick Collins, perhaps one of the earliest to research musical influence recognition, investigated content-based classification of Synth Pop tracks on a small manually annotated dataset of 364 tracks \cite{collins2010computational}. Later he experimented with Prediction by Partial Match (PPM) variable order Markov models, but again the dataset used was relatively small (248 tracks) \cite{collins2012influence}. 

Shalit et al. \cite{shalit2013modeling} presented the first study of musical influence at scale using a topic modeling approach. Specifically, they used the dynamic topic model \cite{blei2006dynamic} and document influence model \cite{gerrish2010language}, time series extensions to traditional topic modeling which allow for the evolution of topics over time. 

With the recent surge in popularity of deep learning based methods, Morton and Kim presented the first application of deep learning to content-based musical influence recognition \cite{morton2015acoustic}. They used a deep belief network for feature extraction from a spectral representation of audio, though they treated influence identification as a multi-label classification problem with only 10 total classes (influencing artists).

\section{Our Contribution}
This thesis explores methods for inferring musical influence relationships through data, focusing primarily on \textit{content-based} methods using song audio. Specifically, first we explore a topic modeling approach to modeling artist-topic influence using the Document Influence Model (DIM), using a larger scale dataset than has been used previously and a bag-of-words feature extraction procedure. Secondly, we present the first (to the best of our knowledge) approach to predicting song-level influence utilizing siamese convolutional neural networks trained on mel-spectogram representations of song audio, achieving an accuracy of $0.7005$. For evaluation of our results, we used as ground truth a network graph of critic-determined influence relationships between musicians scraped from AllMusic.

\section{Thesis Outline}
In the second chapter of this thesis, we detail the various data sources used in this project as well as the methods used to collect that data. The third chapter describes the exploratory analysis conducted in order to investigate the respective feasibilities of both a \textit{network-based} approach, using cover song data from SecondHandSongs, as well as a \textit{content-based} approach, using audio files scraped from AllMusic.

In the fourth chapter, we describe the first content-based approach we tried to model influence, the Document Topic Model. In contrast to previous attempts \cite{shalit2013modeling} to model artist influence using the DIM, we use a larger audio dataset that we scraped from AllMusic.com and a different feature representation than that presented by Shalit et al. Due to the limitations of such an approach, in the fifth chapter we move on to a deep learning strategy using siamese convolutional neural networks for binary classification of artist-to-artist influence. 

Supporting code and data can be found at \url{https://github.com/xueharry/music_influence}.